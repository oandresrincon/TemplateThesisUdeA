%--------------------------------------------------------------------------------------------------------------------------------
\chapter{Chapter title}
%--------------------------------------------------------------------------------------------------------------------------------
\lettrine[lines=2,loversize=0.15,nindent=0pt]{\bf \textcolor{myGris}{T}}{}his chapter demonstrates the method of \index{Gauss-Jordan reduction}Gauss-Jordan reduction in matrices for solving systems of linear equations. Then the operations between matrices and the inverse of a matrix are defined. The chapter ends with algorithms to calculate the (bilateral) inverse and the lateral inverses of a matrix.
%--------------------------------------------------------------------------------------------------------------------------------
% Chapter 1. Section 1.
%--------------------------------------------------------------------------------------------------------------------------------
\section{Definition}
\begin{definition}

A linear equation is an equation of the form $a_1x_1+\dots+a_nx_n=b$ with $a_1,\dots,a_m$ and $b$ real constants and $x_1,\dots,x_n$ the \index{Variables} variables or unknowns from the equation. In this case, the equation is said to be a linear equation in $n$ variables. The numbers $a_1,\dots,a_m$ are called the \index{Coefficients of a linear equation} coefficients of the equation, and the constant $b$ is called the independent term. A system of equations is a finite set of equations. According to this, a system of linear equations is a set of equations of the following form...

\end{definition} 
%--------------------------------------------------------------------------------------------------------------------------------
% Chapter 1. Section 2.
%--------------------------------------------------------------------------------------------------------------------------------
% Example of the example format.
%--------------------------------------------------------------------------------------------------------------------------------
\section{Example}
\begin{example}  
There are several methods to solve this \index{Linear system}linear system, due to the similarity with the method of \index{Gauss-Jordan reduction}Gauss-Jordan reduction that will be exposed in this section, the elimination one stands out. This method uses three basic operations to find...
\end{example}  
%--------------------------------------------------------------------------------------------------------------------------------
% Chapter 1. Section 3.
%--------------------------------------------------------------------------------------------------------------------------------
% Example of the theorem format.
%--------------------------------------------------------------------------------------------------------------------------------
\section{Theorem}
\begin{tcolorbox}
\begin{theorem} \label{T:primerteorema} 
Any matrix, applying Gauss-Jordan reduction, can be brought to a reduced staggered form matrix.
\end{theorem}
\end{tcolorbox}%\vspace{-\baselineskip}
%--------------------------------------------------------------------------------------------------------------------------------
% Chapter 1. Section 4.
%--------------------------------------------------------------------------------------------------------------------------------
% Example of the Proof format.
%--------------------------------------------------------------------------------------------------------------------------------
\section{Proof}
\begin{Proof}
The proof is made by induction on the number of columns of $A$.
\end{Proof}
%--------------------------------------------------------------------------------------------------------------------------------
% Chapter 1. Section 5.
%--------------------------------------------------------------------------------------------------------------------------------
% Example of the corollary format.
%--------------------------------------------------------------------------------------------------------------------------------
\section{Corollary}
\begin{tcolorbox}
\begin{corollary}  
A \index{Linear system}linear system with more variables than equations ($n>m$) never has \index{Unique solution}unique solution.
\end{corollary}%}}
\end{tcolorbox}%\vspace{-\baselineskip}
%--------------------------------------------------------------------------------------------------------------------------------
% Chapter 1. Section 5.
%--------------------------------------------------------------------------------------------------------------------------------
% Example of the remark format.
%--------------------------------------------------------------------------------------------------------------------------------
\section{Observation}
\begin{remark} The converses of the three statements of the previous theorem are also true. The section shows that the \index{Elementary row operation on matrices}elementary row operations are reversible, which allows applying elementary row operations to $A'$ until matrix $A$ is retrieved, which returns the system original.
\end{remark}
\begin{remarkbox}{}
\begin{remark} The converses of the three statements of the previous theorem are also true. The section shows that the \index{Elementary row operation on matrices}elementary row operations are reversible, which allows applying elementary row operations to $A'$ until matrix $A$ is retrieved, which returns the system original.
\end{remark}
\end{remarkbox}
%--------------------------------------------------------------------------------------------------------------------------------
\lipsum[1-6]
\subsection{Programs}

\lipsum[1-6]

\begin{tcolorbox}
\vspace*{-3mm}
\begin{programa}
\begin{lstlisting}
def hello_world():
	print("Hello floating world!")
\end{lstlisting}
\end{programa}%}}
\vspace*{-5mm}
\end{tcolorbox}
